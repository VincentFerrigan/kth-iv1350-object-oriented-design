\documentclass[a4paper]{scrreprt}
\usepackage{fancyhdr}
\pagestyle{fancy}
\usepackage[english]{babel}
\usepackage[utf8]{inputenc}

\usepackage{graphicx}
\usepackage{url}
\usepackage{textcomp}
\usepackage{amsmath}
\usepackage{lastpage}
\usepackage{pgf}
\usepackage{wrapfig}
\usepackage{fancyvrb}
\usepackage{float}

%\usepackage[style=ieee]{biblatex} % om du vill köra ieee (tror leif kör den)
\usepackage[backend=biber,style=vancouver]{biblatex}
\addbibresource{bibfile.bib}
%Alternativt med natbib
%\usepackage[numbers]{natbib} %referenser vancouver
%\usepackage[numbers,round]{natbib} %referenser vancouver
%\usepackage{hyperref}

\usepackage{pdflscape}
% \usepackage{lscape}

% Code highligting
% \usepackage{minted}
\usepackage[outputdir=output/tex]{minted} % iom min makefile



\usepackage[font=footnotesize,labelfont=bf,skip=2pt]{caption}
\usepackage{hyperref}
\newenvironment{longlisting}{\captionsetup{type=listing}}{}
% \renewcommand\listoflistingscaption{Källkod....}
\renewcommand\listoflistingscaption{List of source codes}
\setmintedinline[java]{breaklines=true,breakanywhere=true} % necessary for breakanywhere to work later on.

\usepackage{paralist} %Inline lists

% Create header and footer
% Create header and footer
\headheight 27pt
\pagestyle{fancyplain}
\lhead{\footnotesize{Object-Oriented Design, IV1350}}
\chead{\footnotesize{Seminar 2 Design}}
\rhead{}
\lfoot{}
\cfoot{\thepage\ (\pageref{LastPage})}
\rfoot{}

% Create title page
\title{Seminar 2, Design}
\subtitle{Object-Oriented Design, IV1350}
\author{Vincent Ferrigan ferrigan@kth.se}

\begin{document}

\maketitle

\tableofcontents %Generates the TOC

\chapter{Introduction}

This is the second version of the report for seminar 5, where a more thorough text
is added in Result, and an additional section has been added to the Discussion.
The rest is kept as it was when it was handed in for seminar 2.

The aim of this seminar task is to apply UML and Object-Oriented Patterns when
designing object-oriented architecture and develop object-oriented
systems/programs.

During the task the author has practiced creating both \emph{class diagrams} and
\emph{Interactive diagrams}.
Here the group, consisting of \emph{Elin Blomquist} and \emph{Robert Koivusalo},
has chosen to use \emph{Sequence diagrams} instead of \emph{Communication diagrams}
as Interactive diagrams.

\chapter{Method}
\section{UML}
The UML modeling tool used was \emph{PlantUML}.
PlantUML is an open-source tool for creating UML diagrams using a text-based
syntax.
It was chosen to enable version control through git and to avoid using a
proprietary graphical editor.

UML includes interaction diagrams to illustrate how objects interact via messages.
Two types were introduced in the course textbook; sequence diagrams and communication diagrams.
The group found Sequence diagrams to be more notationally rich and that they had more advantages
over communication diagrams.
More about this choice is discussed in the section \ref{sec:sequence-diagrams-over-communication-diagrams}

\section{The Design Method}
When creating the Class Diagram in figure \ref{fig:CD} we followed the MVC and layer patterns
mentioned in the course textbook,
and chose to use the layers that were used in the course textbooks case study.

Together, we designed one system operation at a time, while striving for \emph{high cohesion},
\emph{low coupling} and a high degree of \emph{encapsulation}.

After each system operation, we updated the class diagrams with the objects and their attributes and methods in use.
\begin{enumerate}
    \item The \verb|StartSale| System Operations -- see figure~\ref{fig:SD}
    \item The \verb|registerItem| System Operations -- see figure~\ref{fig:SD1}
    \item The \verb|endSal|e System Operations -- see figure~\ref{fig:SD2}
    \item The \verb|discountRequest| System Operations -- see figure~\ref{fig:SD3}
    \item The \verb|addPayment| System Operations -- see figure~\ref{fig:SD4}
    \item The Start up, \verb|Main| System Operation -- see figure~\ref{fig:SD5}
\end{enumerate}
Naturally we revised every system operation and discussed which methods and attributes that were to be
public, private or package private and when or if it was appropriate to use \emph{DTOs (Data Transfer Objects)} or not.
All the DTOs that were chosen are depicted in figure~\ref{fig:DTO}.

\newpage
\chapter{Result}
\label{sec:result}
\section{The Class Diagram}
The Class Diagram illustrated in figure \ref{fig:CD} emphasizes the links between the objects, i.e.\ the instances of
our classes.
It also groups the classes into subsystems in accordance with the MVC architectural pattern.

All possible attributes and methods that belong to a class (including their parameters) are shown.
In other words, all methods (and parameters) that were depicted in the Interaction Diagrams are all
collected and grouped here in the Class Diagram.
\begin{figure}[h!]
    \begin{center}
        \includegraphics[width=\textwidth]{./uml/output/CD_Sem1}
        \caption{The class diagram after all system operations have been designed.}
        \label{fig:CD}
    \end{center}
\end{figure}


    \begin{figure}[h!]
        \begin{center}
            \includegraphics[width=\textwidth]{./uml/output/CD_Sem1_001}
            \caption{The DTOs (Data Transfer Objects)}
            \label{fig:DTO}
        \end{center}
    \end{figure}

\newpage
%    \newgeometry{left=0cm,right=0cm,top=0cm,bottom=0cm}
\section{Sequence Diagrams}

Figure ~\ref{fig:SD} shows the Sequence Diagram of the operation startSale
where an instance of Sale is created, which indicates that a new sale is started.

\begin{figure}[h]
    \begin{center}
        \includegraphics[trim=0cm 0cm 51cm 0cm ,clip, width=\textwidth]{./uml/output/SD_Sem2}
%        \caption{The \verb|startSale| system operation }
        \caption{The startSale system operation }
        \label{fig:SD}
    \end{center}
\end{figure}
Figure ~\ref{SD1} is the Sequence Diagram for registerItem,
illustrating the method calls for when the system registers items to be sold,
including the additional flows.
This is done by the cashier entering an item identifier from the view,
that is forwarded to the \mintinline{java}{InventorySystem}.
If the item exists in the \mintinline{java}{InventorySystem} it is added to the sale,
with increased quantity if that item is already in the sale.
If the item does not exist, an error message is sent to the view.
If the item is added to the sale, the total price is updated.

\begin{figure}[h]
    \begin{center}
        \includegraphics[trim=0cm 0cm 40cm 0cm, clip, width=\textwidth]{./uml/output/SD_Sem2_001}
%        \caption{The \verb|registerItem| system operation }
        \caption{The registerItem operation }
        \label{fig:SD1}
    \end{center}
\end{figure}

Figure ~\ref{fig:SD2} shows the operation of endSale,
that is an alert to the system that no more items are to be registered for the current sale,
and the system creates a saleDTO containing all the information about the current sale that is about to be paid for.

\begin{figure}[h]
    \begin{center}
        \includegraphics[trim=0cm 0cm 50cm 0cm, clip, width=\textwidth]{./uml/output/SD_Sem2_002}
%        \caption{The \verb|endSale| system operation }
        \caption{The endSale system operation }
        \label{fig:SD2}
    \end{center}
\end{figure}

Figure ~\ref{fig:SD3} shows the operation discountRequest.
To see if the customer has a right to a discount their costumer identifier and the saleDTO is sent to the DiscountRegister.
If a discount is applied to the sale, the total price is reduced and a new saleDTO, with the updated price, is returned.

\begin{figure}[h]
    \begin{center}
        \includegraphics[trim=0cm 0cm 30cm 0cm, clip, width=\textwidth]{./uml/output/SD_Sem2_003}
        \caption{The discoutRequest system operation }
        \label{fig:SD3}
    \end{center}
\end{figure}

Figure ~\ref{fig:SD4} shows the operation addPayment where the system creates and handles the payment,
registers it in the CashRegister,
logs it in the SaleLog and updates the external systems \mintinline{java}{InventorySystem} and
\mintinline{java}{AccountingSystem} with the information from the sale.
Then a receipt, including the sale information, is printed.

\begin{figure}[h]
    \begin{center}
        \includegraphics[trim=0cm 0cm 0cm 0cm, clip, width=\textwidth]{./uml/output/SD_Sem2_004}
%        \caption{The \verb|addPayment| system operation }
        \caption{The addPayment system operation }
        \label{fig:SD4}
    \end{center}
\end{figure}

Figure ~\ref{fig:SD5} shows the operation startUp, which is the Main.
The diagram illustrates how the system creates instances of other classes; Printer and SaleLog.
Then an instance of Controller is created,
which handles the creation of the CashRegister and external systems;
\mintinline{java}{AccountingSystem}, \mintinline{java}{DiscountSystem} and \mintinline{java}{InventorySystem}.
Lastly, Main creates the View

\begin{figure}[h]
    \begin{center}
        \includegraphics[trim=0cm 0cm 0cm 0cm, clip, width=\textwidth]{./uml/output/SD_Sem2_005}
%        \caption{The Start Up \verb|Main| system operations}
        \caption{The Start Up Main system operations}
        \label{fig:SD5}
    \end{center}
\end{figure}

%    \restoregeometry
\chapter{Discussion}
\label{sec:discussion}
\section{Sequence Diagrams over Communication Diagrams}
\label{sec:sequence-diagrams-over-communication-diagrams}
As mentioned earlier, the group found Sequence diagrams to be more notationally rich and advantageous
over communication diagrams.
For example, it is easier to see the ''call-flow sequence'' with sequence
diagrams -- since they simply read from top
to bottom.
With communication diagrams one must read the sequence numbers which were,
sometimes, hard to follow.

Additionally, the group chose Sequence diagrams over
Communication diagrams for the following reasons:
\begin{enumerate}
    \item Clearer visualization of the timeline: Sequence diagrams provide a clear
    representation of the temporal ordering of interactions, making it easier for
    the team to understand the chronological flow of events in the system.
    \item Better focus on message exchanges: Sequence diagrams place more emphasis on the
    message exchanges between objects, which the group found particularly
    important in a \emph{Point of Sale system}, where understanding the order of interactions can
    help us a great deal when we start programming the system.
    \item More intuitive representation of object lifelines:
    In Sequence diagrams, object
    lifelines are visually clear and make it easy to see when an object is created,
    used, called etc., allowing for a better understanding of object lifecycles
    and their interactions with other objects.
\end{enumerate}

However, naturally, Sequence diagrams have their limitations.
For instance, new objects have to always be added to the right edge, which is
quite limiting—it quickly consumes and exhausts right-edge space on a page.
Free space in the vertical dimension is not efficiently used, potentially
leading to cluttered diagrams and reduced readability.

In conclusion, the group decided to use Sequence diagrams over Communication
diagrams due to their clearer representation of the call-flow sequence,
improved visualization of the timeline and better focus on message exchanges.
Despite their limitations, such as the inefficient use of space,
Sequence diagrams provide more intuitive
object lifeline representation, enhanced readability and maintainability.


%\section{The Start Up Layer}

\section*{Open Issues}
\emph{Which constructors should be package private and which must be public?}
    We chose to make all constructors, other than \mintinline{java}{Receipt} and \mintinline{java}|Amount}, as public.
    There was no need for \mintinline{java}{Receipt} and \mintinline{java}{Amount} to be other than package private since they
were only created by objects withing the same layer and package; the \emph{model}.
    But why not \mintinline{java}{CashPayment}, \mintinline{java}|Item| and \mintinline{java}{Amount}?

    We also had a hard time knowing when, where and how to model the external systems;
\mintinline{java}{DiscountRegister}, \mintinline{java}{AccountingSystem}, \mintinline{java}{InventorySystem}.
    Which attributes should they have?
    Shall we instantiate objects out of them, and if so, when and where?
    It was especially tricky to know which methods and attributes to add to the \mintinline{java}{DiscountRegister}.
    It was hard to read how to implement the business rules.

We chose to put that and the other discussion to rest until we start programming, i.e.\ until the next task.
And hope that much of the issues get clarified during the seminars.

\section*{Added discussion, seminar 5}
While working on this task the group have had several discussions about classes, attributes, and methods.
This lead to a lot of designing and re-designing; for instance there was attributes created to later on be deleted or moved to other classes.
We also created new classes, that was not in the diagrams from seminar 1.
An example of that is ''Amount'' and ''CashPayment''.
In order to make the system more flexible, these are easier to exchange in the future if they are handled as objects in the system.
Another example is the attribute ''quantity'' that initially was an attribute for Sale,
but was removed and added as an attribute to Item instead, since it is the quantity of Item that is handled.
Sale then access ''quantity'' through an list of Items.

There were also a few classes whose being or not being was hard to decide on, such as the class IntegrationHandler.
The purpose of the class IntegrationHandler would have been to make the calls that are currently made by
Controller to the external systems.
This method can be good in larger systems to separate calls to external systems from the actual system functionality.
However, in a system that is as small as ours, it is still a question of not making it over complicated,
and we found it a bit cluttering to add in our diagrams.
Therefore, we decided to not add the IntegrationHandler, and let the Controller handle the external calls instead.

The classes representing the external systems
–- \mintinline{java}{InventorySystem}, \mintinline{java}{AccountingSystem} and \mintinline{java}{DiscountRegister} –- are not
given any attributes since they are not a part of the design of this system.
Calls to them are, however, returning DTOs.
This is because we are expecting that if a call is made to an external system,
it would not send anything but information in this specific case.

The observant reader will, however, detect that the class \mintinline{java}{DiscountDTO} have the two attributes ''customerID''and ''rules''
These attributes were added since that information is provided in the Requirement Specification.
There was discussion about if the DiscountDTO should have attributes at all,
but since it is specified that discount is decided by customerID and rules, we decided to add it, as a point where the system can be further developed.

While making the diagrams, we initially sent Printer as an argument while creating Sale.
In the beginning of the process, this made sense since the printer is used to print the sale information on the receipt.
However, when we were almost done with the diagrams,
we reflected on the matter and decided that it would be better to have it sent from the Controller when it is time to print,
rather than when the sale is initiated, and then moved it accordingly.
This is shown in figure ~\ref{fig:SD4} (1.7 \mintinline{java}{printReceipt(printer:Printer) : void)}.

The creation of Printer and SaleLog was also moved from the Controller to Main.
The further we got designing the system,
it made more sense to have these two instances created in Main instead,
since they are not a part of key system functionality.

While looking at the diagrams, it appears that a lot of the methods are voids and may be package private.
This shows that their purpose is, for example, to change variables within the class.
Methods that are private or package private were not put in the diagrams since they are not yet relevant
for the larger image of system operations.

\end{document}
