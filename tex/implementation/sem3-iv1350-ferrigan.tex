\documentclass[a4paper]{scrreprt}
\usepackage{fancyhdr}
\pagestyle{fancy}
\usepackage[english]{babel}
\usepackage[utf8]{inputenc}

\usepackage{graphicx}
\usepackage{url}
\usepackage{textcomp}
\usepackage{amsmath}
\usepackage{lastpage}
\usepackage{pgf}
\usepackage{wrapfig}
\usepackage{fancyvrb}
%\usepackage{hyperref}

\usepackage{pdflscape}
% \usepackage{lscape}

% Code highligting
% \usepackage{minted}
\usepackage[outputdir=output/tex]{minted} % iom min makefile

\usepackage{caption}
\usepackage{hyperref}
\newenvironment{longlisting}{\captionsetup{type=listing}}{}
% \renewcommand\listoflistingscaption{Källkod....}
\renewcommand\listoflistingscaption{List of source codes}


% Create header and footer
\headheight 27pt
\pagestyle{fancyplain}
\lhead{\footnotesize{Object-Oriented Design, IV1350}}
\chead{\footnotesize{Seminar 3 Implementation}}
\rhead{}
\lfoot{}
\cfoot{\thepage\ (\pageref{LastPage})}
\rfoot{}

% Create title page
\title{Seminar 3, Implementation}
\subtitle{Object-Oriented Design, IV1350}
\author{Vincent Ferrigan ferrigan@kth.se}

\begin{document}

\maketitle

\tableofcontents %Generates the TOC

\chapter{Introduction}
The aim of this seminar task is to %practice basic object-oriented design and
program and test a system under development based on a design that development
during previous seminars.

UML and Object-Oriented Patterns were applied when designing the object-oriented
architecture of this object-oriented system.

%TODO Lägg till github konto.

The author collaborated with
\emph{Elin Blomquist} and \emph{Robert Koivusalo},



\chapter{Method}
\section*{Tools}
The UML modeling tool used was \emph{PlantUML}.
PlantUML is an open-source tool for creating UML diagrams using a text-based
syntax. It was chosen to enable version control through git and to avoid using a
proprietary graphical editor.

All code was written in \emph{IntelliJ IDEA} and the unit tests were written with the \emph{JUnit 5} test framework.

\section*{The overall Work-flow}
Both the \emph{Class Diagram} and \emph{System Diagram}, that were designed during the previous seminars,
provided a visual representation of the system's architecture and design.

This diagram served as a blueprint for the programming process,
guiding the implementation of each system operation.
The programming process was carried out incrementally,
with the focus on implementing one system operation at a time.
This iterative methodology forced the team to constantly review and
update the system design when needed.
Discrepancies were discovered when certain methods
were implemented  and
quite a few issues arose during and were addressed after each system operations
were implemented.

As a result, the system design was continuously refined,
allowing for a more robust and correct system.

%Methods that were too long were split and large classes reviewed and divided into several if and when appropriate.
%The latter in order to keep a higher cohesion.

\section*{The program}
\subsection*{Main}
We started by setting up all the packages (representing the system's layers)
and creating all the classes (including their initial attributes and methods)
according to our UML diagrams.

Overall, the program
contains the layers
\verb|startup|, \verb|view|, \verb|controller|, \verb|model|,
\verb|integration|, and \verb|data|.

As mentioned, we quickly discovered that the initial design had flaws.
following the UML from the previous seminar.
The sequence diagrams, class diagrams and code therefore rewritten and
developed in parallel, and we therefore added functionality gradually to solve and cover all
the discovered issues and blind spots.
During the development process, not only did new problems arise,
but also new ideas on how the program should work to function as intended.

Methods that became very long were divided into smaller methods and classes that became
too big were split in accordance to with the main goal of achieving high cohesion.

\subsection*{The Data Layer}
Instead of hardcoding data in the integration classes, flat file databases were used to store records.
These CSV-files were used to store, read and update data for accounting, inventory, item registry and discounts.

Since the system was reading and writing to files, handling exceptions became necessary.
This was done with \emph{try-catch blocks}.

\section*{Tests}
The tests were created using the JUnit 5 framework
and follow best practice i.e. they have the same directory architecture as the program,
and are placed outside the source folder for the program (the SUT).
We started by creating tests for the controller and item.
The controller covers all system operations
and handles calls to all other layers.
However, the tests in the controller are
much more complex than the tests performed in classes with low dependency.
Items have lower dependency, making the tests much less complex.
The choice of
our tests has led us to cover large parts of the program's functionality.

Testing erroneous conditions was postponed until error handing is added.
Which will be done prior to the next seminar.
%
%Tests were written in bottom-up order, first for c.....
%
%\begin{longlisting}
%    \inputminted[
%        label=Uppgift A,
%        linenos=true,
%        bgcolor=lightgray,
%%        frame=single,
%        fontsize=\footnotesize,
%    ]{java}{../../src/main/se/kth/iv1350/startup/Main.java}
%    \caption{Caption text av något slag.}
%    \label{listing:main}
%\end{longlisting}

%\section{The refined Design}


%When creating the Class Diagram in figure \ref{fig:CD} we followed the MVC and layer patterns
%mentioned in the course textbook,
%and chose to use the layers that were used in the course textbooks case study.
%
%Together, we designed one system operation at a time, while striving for \emph{high cohesion},
%\emph{low coupling} and a high degree of \emph{encapsulation}.
%
%After each system operation, we updated the class diagrams with the objects and their attributes and methods in use.
%\begin{enumerate}
%    \item The \verb|StartSale| System Operations -- see figure~\ref{fig:SD}
%    \item The \verb|registerItem| System Operations -- see figure~\ref{fig:SD1}
%    \item The \verb|endSal|e System Operations -- see figure~\ref{fig:SD2}
%    \item The \verb|registerCustomerToSale| System Operations -- see figure~\ref{fig:SD3}
%    \item The \verb|addPayment| System Operations -- see figure~\ref{fig:SD4}
%    \item The Start up, \verb|Main| System Operation -- see figure~\ref{fig:SD5}
%\end{enumerate}
%Naturally we revised every system operation and discussed which methods and attributes that were to be
%public, private or package private and when or if it was appropriate to use \emph{DTOs (Data Transfer Objects)} or not.
%All the DTOs that were chosen are depicted in figure~\ref{fig:DTO}.

\newpage
\chapter{Result}
\label{sec:result}
The program starts in the layer \verb|StartUp|,
where \verb|Main| is located.
Within \verb|Main|, instances of \verb|Printer|,
\verb|Display|, and \verb|RegisterCreator| are created.
Subsequently, a Controller instance is created using these previously established instances as input arguments.
The Main method then proceeds to create a View instance,
with the Controller passed as an argument.
Following this,
the View (representing the cashier or, in this specific case, the method "hardkodadegrejer")
initiates calls to the Controller.

The entire program can be found here:

\subsubsection*{GitHub-Repo}
\url{https://github.com/VincentFerrigan/kth-iv1350-object-oriented-design}

%\section{The Class Diagram}
%\begin{figure}[h!]
%    \begin{center}
%        \includegraphics[width=\textwidth]{uml/output/CD_Sem1}
%        \caption{The class diagram after all system operations have been designed.}
%        \label{fig:CD}
%    \end{center}
%\end{figure}
%The Class Diagram illustrated in figure \ref{fig:CD} emphasizes the links between the objects, i.e.\ the instances of
%our classes.
%It also groups the classes into subsystems in accordance with the MVC architectural pattern.
%
%All possible attributes and methods that belong to a class (including their parameters) are shown.
%In other words, all methods (and parameters) that were depicted in the Interaction Diagrams are all
%collected and grouped here in the Class Diagram.
\newpage
%\begin{landscape}
    \section*{The Refined Design}
    \subsection*{Class Diagram}
%    \newgeometry{left=0cm,right=0cm,top=0cm,bottom=0cm}

    \begin{figure}[h!]
        \begin{center}
            \includegraphics[width=\textwidth]{../../uml/output/uml_v3}
            \caption{The Class Diagram}
            \label{fig:CD}
        \end{center}
    \end{figure}

    \begin{figure}[h!]
        \begin{center}
            \includegraphics[width=\textwidth]{../../uml/output/uml_v3_001}
            \caption{The DTOs (Data Transfer Objects)}
            \label{fig:DTO}
        \end{center}
    \end{figure}

\newpage
\subsection*{System Operations}
\subsubsection*{The Start up, Main System Operation}
\begin{figure}[h!]
    \begin{center}
        \includegraphics[width=\textwidth]{../../uml/output/uml_v3_002}
%        \includegraphics[trim=0cm 0cm 0cm 0cm, clip, width=\textwidth]{uml/output/uml_v3.png}
%        \caption{The Start Up \verb|Main| system operations}
        \caption{The Start Up Main system operations}
        \label{fig:SD5}
    \end{center}
\end{figure}

\subsubsection*{The StartSale System Operations}
\begin{figure}[h!]
    \begin{center}
        \includegraphics[width=\textwidth]{../../uml/output/uml_v3_003}
%        \includegraphics[trim=0cm 0cm 51cm 0cm ,clip, width=\textwidth]{../uml/output/uml_v3.png}
%        \caption{The \verb|startSale| system operation }
        \caption{The startSale system operation }
        \label{fig:SD}
    \end{center}
\end{figure}
%
\newpage
\subsubsection*{The registerItem System Operations}
\begin{figure}[h!]
    \begin{center}
        \includegraphics[width=\textwidth]{../../uml/output/uml_v3_004}
%        \includegraphics[trim=0cm 0cm 40cm 0cm, clip, width=\textwidth]{uml/output/uml_v3.png}
%        \caption{The \verb|registerItem| system operation }
        \caption{The registerItem operation }
        \label{fig:SD1}
    \end{center}
\end{figure}

\subsection*{Displaying the ouput of current sale}
\begin{longlisting}
    \inputminted[
        label= Output of a current sale,
        linenos=false,
%        bgcolor=lightgray,
        firstline=1,
        lastline=8,
%        frame=single,
        fontsize=\footnotesize,
    ]{bash}{output/outputs.txt}
    \caption{The Receipt printed to System.out}
    \label{listing:checkoutOutput}
\end{longlisting}

\subsubsection*{The endSale System Operations}
\begin{figure}[h!]
    \begin{center}
        \includegraphics[width=\textwidth]{../../uml/output/uml_v3_005}
%        \includegraphics[trim=0cm 0cm 50cm 0cm, clip, width=\textwidth]{uml/output/uml_v3.png}
%        \caption{The \verb|endSale| system operation }
        \caption{The endSale system operation }
        \label{fig:SD2}
    \end{center}
\end{figure}

\subsubsection*{End of Sale - Displaying the output of a checkout}
\begin{longlisting}
    \inputminted[
        label= Output of a checkout,
        linenos=false,
%        bgcolor=lightgray,
        firstline=61,
        lastline=72,
%        frame=single,
        fontsize=\footnotesize,
    ]{bash}{output/outputs.txt}
    \caption{The checkout shopping cart printed to System.out}
    \label{listing:checkout}
\end{longlisting}

\subsubsection*{The discountRequest System Operations}
\begin{figure}[h!]
    \begin{center}
        \includegraphics[width=\textwidth]{../../uml/output/uml_v3_006}
%        \includegraphics[trim=0cm 0cm 30cm 0cm, clip, width=\textwidth]{uml/output/uml_v3.png}
        \caption{The discoutRequest system operation }
        \label{fig:SD3}
    \end{center}
\end{figure}

\subsubsection*{The addPayment System Operations}
\begin{figure}[h!]
    \begin{center}
        \includegraphics[width=\textwidth]{../../uml/output/uml_v3_007}
%        \includegraphics[trim=0cm 0cm 0cm 0cm, clip, width=\textwidth]{uml/output/uml_v3.png}
%        \caption{The \verb|addPayment| system operation }
        \caption{The addPayment system operation }
        \label{fig:SD4}
    \end{center}
\end{figure}

\subsubsection*{Receipt}
\begin{longlisting}
    \inputminted[
        label= Receipt,
        linenos=false,
%        bgcolor=lightgray,
        firstline=73,
        lastline=91,
%        frame=single,
        fontsize=\footnotesize,
    ]{bash}{output/outputs.txt}
    \caption{The Receipt printed to System.out}
    \label{listing:receipt}
\end{longlisting}

%    \restoregeometry
%\end{landscape}


\chapter{Discussion}
\label{sec:discussion}

During the development
process, one constantly encounters new problems that need to be solved.
We wanted to create a program with as little hard coding as possible, and since we
haven't learned how to implement databases properly yet,
the obvious choice was to use CSV files as flat file databases.
We were careful with the documentation and made sure that all
declarations have explanatory names and follow convention.

\section*{Flexibility and efficiency}
To easily change the currency, depending on which region the
system will be used in, we used both Javas currency class and a Locale class.
The class provides
methods for obtaining information such as language, country code, and currency
by region.
By changing the region, the unit of the shopping cart and receipt
amount changes to display, for example, €, £, etc.

In \verb|addItem|, we check if the
item is already in the shopping cart and do not need to fetch item information
again from the inventory.
This makes the program faster and more efficient, as
you do not have to retrieve the same type of information repeatedly

When creating systems and their tests, they should be made as flexible as possible
and made easy to change
when needed.
Which we believe that we have.

\section*{The code does not smell}
\emph{Code smell} refers to characteristics or patterns within a program's source code that
indicate potential issues, often making the code difficult to maintain, understand, or extend.
Eventhough these smells might not immediate errors, they may lead to technical debt,
decreased software quality, and more complex debugging processes in the long run.

We kept functions and classes smal and focused, regularly refactored code and wrote descriptive names for
variables, functions and classes.
Code duplication was avoided, \emph{JavaDocs} was written for all but private functions.

\section*{Testing}
\label{sec:testing}
Writing tests before or immediately after implementing a method would have helped us identify
potential issues early on and ensured code quality.
In other words, a Test-Driven-Development approach would have been most desirable.
Unfortunately, the entire program was written before writing a single test.

While implementing the JUnit tests, the team and I quickly understood that
the optimal way would have been to apply the Test-Driven-Development approach.
It would have both saved us time, and improved our design and overall code quality.

We have created the tests to be self-evaluating and they display an informative message
if a test should fail.
We have also followed the correct structure so that the
tests are not located under the source folder for the program.

Not all tests have been written yet, which is something I intend to do prior to the next seminar.

%As mentioned earlier, the group found Sequence diagrams to be more notationally rich and advantageous
%over communication diagrams.
%For example, it is easier to see the ''call-flow sequence'' with sequence
%diagrams -- since they simply read from top
%to bottom.
%With communication diagrams one must read the sequence numbers which were,
%sometimes, hard to follow.
%
%Additionally, the group chose Sequence diagrams over
%Communication diagrams for the following reasons:
%\begin{enumerate}
%    \item Clearer visualization of the timeline: Sequence diagrams provide a clear
%    representation of the temporal ordering of interactions, making it easier for
%    the team to understand the chronological flow of events in the system.
%    \item Better focus on message exchanges: Sequence diagrams place more emphasis on the
%    message exchanges between objects, which the group found particularly
%    important in a \emph{Point of Sale system}, where understanding the order of interactions can
%    help us a great deal when we start programming the system.
%    \item More intuitive representation of object lifelines:
%    In Sequence diagrams, object
%    lifelines are visually clear and make it easy to see when an object is created,
%    used, called etc., allowing for a better understanding of object lifecycles
%    and their interactions with other objects.
%\end{enumerate}
%
%However, naturally, Sequence diagrams have their limitations.
%For instance, new objects have to always be added to the right edge, which is
%quite limiting—it quickly consumes and exhausts right-edge space on a page.
%Free space in the vertical dimension is not efficiently used, potentially
%leading to cluttered diagrams and reduced readability.
%
%In conclusion, the group decided to use Sequence diagrams over Communication
%diagrams due to their clearer representation of the call-flow sequence,
%improved visualization of the timeline and better focus on message exchanges.
%Despite their limitations, such as the inefficient use of space,
%Sequence diagrams provide more intuitive
%object lifeline representation, enhanced readability and maintainability.
%
%
%%\section{The Start Up Layer}
%
%\section{Open Issues}
%\label{sec:open_issues}
%\emph{Which constructors should be package private and which must be public?}
%    We chose to make all constructors, other than \verb|Receipt| and \verb|Amount|, as public.
%    There was no need for \verb|Receipt| and \verb|Amount| to be other than package private since they
%were only created by objects withing the same layer and package; the \emph{model}.
%    But why not \verb|CashPayment|, \verb|ShoppingCartItem| and \verb|Amount|?
%
%    We also had a hard time knowing when, where and how to model the external systems;
%\verb|CustomerRegistry|, \verb|AccountingSystem|, \verb|InventorySystem|.
%    Which attributes should they have?
%    Shall we instantiate objects out of them, and if so, when and where?
%    It was especially tricky to know which methods and attributes to add to the CustomerRegistry.
%    It was hard to read how to implement the business rules.
%
%
%We chose to put that and the other discussion to rest until we start programming, i.e.\ until the next task.
%And hope that much of the issues get clarified during the seminars.

\listoflistings % Now typeset the list
\end{document}
