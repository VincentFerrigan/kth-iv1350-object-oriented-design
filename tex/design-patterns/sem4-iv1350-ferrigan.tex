\documentclass[a4paper]{scrreprt}
\usepackage{fancyhdr}
\pagestyle{fancy}
\usepackage[english]{babel}
\usepackage[utf8]{inputenc}

\usepackage{graphicx}
\usepackage{url}
\usepackage{textcomp}
\usepackage{amsmath}
\usepackage{lastpage}
\usepackage{pgf}
\usepackage{wrapfig}
\usepackage{fancyvrb}
\usepackage{float}

%\usepackage[style=ieee]{biblatex} % om du vill köra ieee (tror leif kör den)
\usepackage[backend=biber,style=vancouver]{biblatex}
\addbibresource{bibfile.bib}
%Alternativt med natbib
%\usepackage[numbers]{natbib} %referenser vancouver
%\usepackage[numbers,round]{natbib} %referenser vancouver
%\usepackage{hyperref}

\usepackage{pdflscape}
% \usepackage{lscape}

% Code highligting
% \usepackage{minted}
\usepackage[outputdir=output/tex]{minted} % iom min makefile



\usepackage[font=footnotesize,labelfont=bf,skip=2pt]{caption}
\usepackage{hyperref}
\newenvironment{longlisting}{\captionsetup{type=listing}}{}
% \renewcommand\listoflistingscaption{Källkod....}
\renewcommand\listoflistingscaption{List of source codes}
\setmintedinline[java]{breaklines=true,breakanywhere=true} % necessary for breakanywhere to work later on.

\usepackage{paralist} %Inline lists

% Create header and footer
\headheight 27pt
\pagestyle{fancyplain}
\lhead{\footnotesize{Object-Oriented Design, IV1350}}
\chead{}
\rhead{\footnotesize{Seminar 4 Exceptions and Design Patterns}}
\lfoot{}
\cfoot{\thepage\ (\pageref{LastPage})}
\rfoot{}

% Create title page
\title{Seminar 4, Exceptions and Design Patterns}
\subtitle{Object-Oriented Design, IV1350}
\author{Vincent Ferrigan ferrigan@kth.se}

\begin{document}

\maketitle

\tableofcontents %Generates the TOC

\chapter{Introduction}
The purpose of this report is to discuss the implementation of exception handling
and design patterns.
The aim of the fourth seminar was on practicing the design
and coding of exception handling, polymorphism, and design patterns,
specifically as outlined in chapters eight and nine of the course book.

The seminar consisted of two tasks.
The first task was to use both checked and
unchecked exception handling.
The former to manage the alternative flow, step $3-4a$
in the Process Sale, and the latter to indicate database failure (such as a
database server not running or database connectivity lost).
All caught exceptions were to be logged for developers, and those caught
in \verb|View|, an informative message displayed to the user.

The second task also consisted of two parts.
The first part, \emph{task 2a}, revolved
around employing the Observer Design Pattern to introduce a new functionality:
displaying the total revenue from the start of the program, each time it was
updated.
This functionality was implemented through the creation of two new
classes.
The first called \mintinline{java}{TotalRevenueView}, responsible for displaying the
total income and a second class called \mintinline{java}{TotalRevenueFileOutput} for logging the
income to a file.

The second part of the second task, \emph{task 2b}, involved refactoring other parts of
the code to two or more GoF patterns.

In total, the following eight GoF design patterns are implemented:
\begin{enumerate}
    \item \textbf{The Creational Patterns}
    \begin{enumerate}
        \item \emph{Singleton},
        \item \emph{Abstract Factory} and
        \item \emph{Factory method}.
    \end{enumerate}
    \item \textbf{The Structural Patterns}
    \begin{enumerate}
        \setcounter{enumi}{3}
        \item \emph{Composite} and
        \item \emph{Facade}.
    \end{enumerate}
    \item \textbf{The Behavioral Patterns}
    \begin{enumerate}
        \setcounter{enumi}{5}
        \item \emph{Observer},
        \item \emph{Strategy} and
        \item \emph{Template Method}.
    \end{enumerate}
\end{enumerate}

\chapter{Method}
\section*{Tools}
This project was implemented in \emph{Java}.
The UML modeling tool used was \emph{PlantUML}.
PlantUML is an open-source tool for creating UML diagrams using a text-based
syntax.
It was chosen to enable version control through git and to avoid using a
proprietary graphical editor.
This report was also written in plain text mode -- \LaTeX.

All code was written in \emph{IntelliJ IDEA} and the unit tests were written with the \emph{JUnit 5} test framework.
Quick-fixes and editing was, however, done in \emph{Vim}.

\section*{The overall Work-flow}
Incremental refactoring, one feature at a time.
Automatic tests were run to make sure that the changes didn't break the code.

To improve the design and implement certain design patterns,
one had to continuously analyze the code-base to identify code smells and
(re)valuate parts of the code that can be refactored to a
particular design pattern (or patterns).
Several resources were used; the course textbook, The GoF Design Patterns
and Clean Code -- A handbook of Agile Software Craftmanship.
The latter was mostly used as a guide in implementing error and exception handling.

As a result, the system design was continuously refined,
allowing for more clean and maintainable code and a greater ease to add new code.

\section*{The project's resources and the system's configuration and data}
\label{sec:resources}
Instead of hard-coding data in the integration classes,
flat file databases are used to store records.
These CSV-files store,
read and update data for accounting, inventory (a product catalog included), and customer details.

The filenames and their paths are all found in the project's \verb|Config.Properties|
file under \verb|src/main/resources|.
The class names of the objects a \emph{Factory} can create, and the filenames of all the loggers are also
configured in the same file.

The properties are added to the \verb|System Properties| -- a set-up based on
\href{https://docs.oracle.com/javase/tutorial/essential/environment/sysprop.html}{The Java\texttrademark{} Tutorials - System Properties}

If a developer is having trouble loading the resource file \verb|config.properties|,
they will have to first check that the \verb|src/main/resources|
is correctly configured as a resource directory in your IDE.

Since the system reads and writes to files, handling exceptions became necessary.

\section*{Tests}
Tests are created using the JUnit 5 framework and follow best practice, i.e., they
have the same directory architecture as the program, and are placed outside the source
folder for the program (the SUT).

Similar to the SUT, the tests have their own Config Properties file, error-log file and flat file databases
(see listing ~\ref{listing:test-config.properties}) and are added to the \verb|System Properties| with
a \mintinline{java}{@BeforeAll} method (see listing ~\ref{listing:before-all-tests-setup}).

In this way, the inventory can be changed on the fly, just by manipulating
the \verb|inventory_items.csv| file which is placed \verb|test-data/db| directory.
The test can create, compare and manipulate its own database and log-files.
This enables testing classes in both the integration and util layer package.
. %TYP TODO men har du gjort så? Lägg till ErrorFileLogHandler test, TotalRevenueFileOutput test.

\begin{longlisting}
    \inputminted[
        label=@BeforeAll tests setup,
        linenos=true,
        bgcolor=lightgray,
        firstline=1,
        lastline=25,
%        frame=single,
        fontsize=\footnotesize,
    ]{bash}{../../src/test/resources/config.properties}
    \caption{The Config Properties file for the JUnit tests.}
    \label{listing:test-config.properties}
\end{longlisting}

\newpage
\begin{longlisting}
    \inputminted[
        label=@BeforeAll tests setup,
        linenos=true,
        bgcolor=lightgray,
        firstline=10,
        lastline=34,
%        frame=single,
        fontsize=\footnotesize,
    ]{java}{../../src/test/se/kth/iv1350/POSTestSuperClass.java}
    \caption{The Config Properties are added to the System Properties in a @BeforeAll setup method.
    This method is placed in a superclass, which is extended in all test that require a pre-configruation}
    \label{listing:before-all-tests-setup}
\end{longlisting}

\newpage
\chapter{Result}
\label{sec:result}
The entire program can be found here on GitHub:

\subsubsection*{GitHub-Repo}
\url{https://github.com/VincentFerrigan/kth-iv1350-object-oriented-design}

\section*{Task 1 - Exceptions}
\emph{Checked exceptions} are usually used for error handling business logic, while
\emph{unchecked exceptions} are used for programming errors.
To manage the alternative flow, step $3-4a$,
where an item is not found in the item registry,
a checked exception named \mintinline{java}{ItemNotFoundException} is used.
It is thrown when an item identifier is not valid or is not found in the Item Registry.
\mintinline{java}{ItemNotFoundException} extends \mintinline{java}{Exception}
and creates a message to inform the user that there is no
item with the item identifier that was provided to the system, along with the incorrect item
identifier.
This exception is then passed through the system and is caught in the View, where it is
handled.
The user gets notified by an error message.

The exception is declared in the signature of each method between the throw from the
integration layer %(Item Registry)
to the catch in view.

Error handling when records are not found in the Customer Registry is
also implemented with
a checked exception, \mintinline{java}{CustomerNotFoundIncustomerRegistryException},
and handled in the same fashion as above.

The program also handles I/O exceptions when reading and writing to the flat file databases.

A database failure can be simulated by registering the item id $404$,
which triggers the ItemRegistry (a class that represents an external inventory system)
to throw the program's unchecked exception \mintinline{java}{ItemRegistryException}.

Additional unchecked exceptions are implemented and handled.
\begin{enumerate}
    \item \mintinline{java}{CustomerRegistryException} and \mintinline{java}{AccountingSystemException} for
    failures related to the other two databases.
    \item \mintinline{java}{PricingFailureException} and \mintinline{java}{RegistryHandlerException}.
    Both are used when reflective operation exceptions are caught.
    The former is the programs own unchecked exception for pricing strategies
    (i.e. Factory-Method failure) while the latter
    is the programs unchecked exception for
    when the program joint database interface fasad fails to set up the registries
    (i.e. Abstract Factory failure).
    \item \mintinline{java}{IllegalStateException} is thrown if the
    system operations are called in the wrong order.
    For example, when registering an item before initializing a sale
    or calling the pay operation before ending sale.
\end{enumerate}

All the above-mentioned unchecked exceptions %for registry, pricing strategies and fasad
are caught and re-thrown as the program's own checked
exception \mintinline{java}{OperationFailedException}
by the \verb|Controller|.
It is in this try-catch statement that the developer gets notified by an error-log.
This exception is then caught in the \verb|View|, where it is
handled.
The user gets notified by an error message.

\mintinline{java}{ItemNotFoundException} is manually tested by the sample method
\mintinline{java}{alternativeFlow3AWithCheckedExceptions()}
in \verb|View|
and automatically tested in the unit test classes \verb|SaleTest| and \verb|ControllTest|.
The output of this alternative flow is demonstrated in
\href{https://github.com/VincentFerrigan/kth-iv1350-object-oriented-design#alternative-flow---checked-exception-business-logic}{Alternative-flow-3a-gif}
and illustrated in
figure ~\ref{fig:reg-item}, ~\ref{fig:item-not-in-shopping-cart}, ~\ref{fig:notify-sale-observers} and ~\ref{fig:notify-users-developers}.

\mintinline{java}{ItemRegistryException} and \mintinline{java}{OperationFailedException}
are tested manually
by the sample method \mintinline{java}{basicFlowWithUnCheckedExceptions()}
in \verb|View| and are automatically tested in the unit test
classes \verb|SaleTest| and \verb|ControllTest|.
The output of this alternative flow is demonstrated in
\href{https://github.com/VincentFerrigan/kth-iv1350-object-oriented-design#alternative-flow---unchecked-exception}{Unchecked-exception-gif}
and also illustrated by
figure ~\ref{fig:reg-item}, ~\ref{fig:item-not-in-shopping-cart}, ~\ref{fig:notify-sale-observers} and ~\ref{fig:notify-users-developers}.

\subsubsection{Logging exceptions and notifying users and developers}
Both the \mintinline{java}{ErrorMessageHandler} and \mintinline{java}{ErrorFileLogHandler}
are \emph{Singletons} that implement the \mintinline{java}{Logger} interface.
They create a message from the exceptions
caught.
\mintinline{java}{ErrorMessageHandler}
is responsible for showing error messages to the user.
It's a dummy implementation that prints to
\mintinline{java}{System.out} while
\mintinline{java}{ErrorMessageHandler}
is responsible for logging errors to a specific file.
This process is demonstrated in
\href{https://github.com/VincentFerrigan/kth-iv1350-object-oriented-design#alternative-flow---unchecked-exception}{Alternative-flow-unchecked-exception-gif}
and illustrated in
figure ~\ref{fig:reg-item} and ~\ref{fig:notify-users-developers}

\newpage
\section*{Task 2 - Refactoring to Patterns}
\subsubsection{The observer and template method design pattern}
The Observer design-pattern is applied not only for the task at hand (revenue updates)
but also for displaying sale details.
The display is updated each time an item is added to the shopping cart
and during end-of-sale.
The latter is performed in order to display possible discounts.

The outcome of refactoring with the observer-pattern
is demonstrated in all samples included in the README file,
and therefore presented on the GitHub repository page.
How the observers are added to the observables are illustrated both
in figure ~\ref{fig:start-up} and ~\ref{fig:start-sale}.
The sale observers are notified in \mintinline{java}{Sale} by the
\mintinline{java}{addItem()} and
\mintinline{java}{endSale()} method
(see figure ~\ref{fig:reg-item}, ~\ref{fig:end-sale}
and ~\ref{fig:notify-sale-observers}),
while the cash register observers are notified in
\mintinline{java}{CashRegister} by the
\mintinline{java}{addPayment()} method
(see figure \ref{fig:pay} and ~\ref{fig:notify-cashregister-observer}).
The former displays sale details (including the shopping cart)
and the latter displays the total revenue
when it has changed.

Since the observers perform similar steps in the same order,
the Template Method design-pattern is used.
The abstract class \mintinline{java}{SaleView} is the template
for sale observers
while the abstract class \mintinline{java}{TotalRevenue}
acts as the template for cash register observers
(see the class diagram in figure ~\ref{fig:CD} for further details).
%duplicated code was eliminated by using the Template Method Pattern.

The observers are only exposed to certain methods or attributes.
For example, a limited view of a particular sale is sent to the
sale observers by using an ''observered-interface'' called
\mintinline{java}{LimitedSaleView}.
This interface is implemented by the wrapper class
\mintinline{java}{LimitedSaleViewWrapper} that
only exposes certain Sale methods.

All logging systems in this program, which includes the observer that logs the total revenue to a file,
should only have one instance of the logger,
as multiple loggers can cause inconsistent and cluttered logs.
The Singleton pattern is therefore applied to create a
single logger instance.
This instance can be accessed from different parts of the code.

\subsubsection{The facade as interface for a set of registry interfaces}
The previous RegisterCreator (see previous seminars) was refactored to Facade.
A facade provides a unified interface to a
set of interfaces in a subsystem [GoF, "Design Patterns"].
Here it represents a ''service container'' where the Facade Design pattern was applied.
This service container is responsible for instantiating all registers
(external systems/databases) and provides
a unified and simplified interface for the client classes.

The Abstract Factory was used with Facade to provide an
interface for creating registry objects
(see \mintinline{java}{IRegistryFactory}
and \mintinline{java}{FlatFileDatabaseFactory} in figure ~\ref{fig:facade}

Both the facade and the registries are implemented as
Singletons since only one facade per subsystem is needed.
Please note howevever, that the \mintinline{java}{SaleLog}
is not implemented as a Singleton.

%a single instance available to all of the program, for example, a single database object shared by different parts of the program.g

\subsubsection{Discount and promotion algorithms - a combination of Strategy and Composite Patterns}
Conditional discount logic can be replaced with
a combination of the Strategy and Composite Pattern.
Here, each discount (or promotion) algorithm
can be defined in a separate strategy-class, but with a common interface
(see listing ~\ref{listing:discount-strategy}).

Knowing, however, which discount strategy to choose from
can be solved with the composite design-pattern.
This ''Composite-Discount-Strategy''
(shown in the listing ~\ref{listing:discount-composite})
performs multiple discount and promotion algorithms but chooses only
one of them -- the one that returns the lowest total price.

\subsubsection*{Adding or substituting discounts - The Factory Method}
The Factory Method is applied in order to enable the developers (or users) to substitute
or add discounts.
Here even the creational knowledge is moved to the factory,
since discounts and promotions can be substituted or added at runtime (and after deployment).
The Discount Factory class is shown in listing ~\ref{listing:discount-factory}.

The class names of the objects a \emph{Factory} can create, 
can be added in the in the project's \verb|Config.Properties| 
file under \verb|src/main/resources| 
(see section \ref{sec:resources} on project resources and systems configuration).

\begin{longlisting}
    \inputminted[
        label=The common Discount-Strategy Interface,
        linenos=true,
        bgcolor=lightgray,
        firstline=11,
        lastline=21,
%        frame=single,
        fontsize=\footnotesize,
    ]{java}{../../src/main/se/kth/iv1350/integration/pricing/DiscountStrategy.java}
    \caption{Discount strategy defining the ability to calculate total price.
        This interface shall be implemented by a class providing a promotion or
        discount algorithm.}
    \label{listing:discount-strategy}
\end{longlisting}

\begin{longlisting}
    \inputminted[
        label=The Compostie Discount-Strategy,
        linenos=true,
        bgcolor=lightgray,
        firstline=16,
        lastline=70,
%        frame=single,
        fontsize=\footnotesize,
    ]{java}{../../src/main/se/kth/iv1350/integration/pricing/CompositeDiscountStrategy.java}
    \caption{A DiscountStrategy, which performs multiple discount and promotion algorithms.
        All these algorithms that are added to this composite are executed,
        and the lowest total price is found.
        }
    \label{listing:discount-composite}
\end{longlisting}

\begin{longlisting}
    \inputminted[
        label=The Discount-Factory,
        linenos=true,
        bgcolor=lightgray,
        firstline=10,
        lastline=73,
%        frame=single,
        fontsize=\footnotesize,
    ]{java}{../../src/main/se/kth/iv1350/integration/pricing/DiscountFactory.java}
    \caption{A Discount Factore implmented as a
        Singleton that creates instances of the algorithm used for calculating total price with
        discounts or promotions.
        All such instances must implement DiscountStrategy.}
    \label{listing:discount-factory}
\end{longlisting}

\section*{Output-samples}
\subsubsection*{Basic flow}
\href{https://github.com/VincentFerrigan/kth-iv1350-object-oriented-design/tree/main#basic-flow}{Basic-flow-gif}

\subsection*{Alternative Flows}
\subsubsection*{Alternative Flow -- Same Item Identification Number}
\href{https://github.com/VincentFerrigan/kth-iv1350-object-oriented-design#alternative-flow---same-item-identification-number}{Alternative-flow-3b-gif}

\subsubsection*{Alternative Flow - Multiple Items of The Same Goods}
\href{https://github.com/VincentFerrigan/kth-iv1350-object-oriented-design#alternative-flow---multiple-items-of-the-same-goods}{Alternative-flow-3c-gif}

\subsection*{Exceptions}
\subsubsection*{Alternative Flow -- Checked Exception, Business Logic}
\href{https://github.com/VincentFerrigan/kth-iv1350-object-oriented-design#alternative-flow---checked-exception-business-logic}{Alternative-flow-3a-gif}

\subsubsection*{Alternative Flow -- Unchecked Exception}
\href{https://github.com/VincentFerrigan/kth-iv1350-object-oriented-design#alternative-flow---unchecked-exception}{Unchecked-exception-gif}

%\section{The Class Diagram}
%\begin{figure}[h!]
%    \begin{center}
%        \includegraphics[width=\textwidth]{../../uml/output/CD_Sem1}
%        \caption{The class diagram after all system operations have been designed.}
%        \label{fig:CD}
%    \end{center}
%\end{figure}
%The Class Diagram illustrated in figure \ref{fig:CD} emphasizes the links between the objects, i.e.\ the instances of
%our classes.
%It also groups the classes into subsystems in accordance with the MVC architectural pattern.
%
%All possible attributes and methods that belong to a class (including their parameters) are shown.
%In other words, all methods (and parameters) that were depicted in the Interaction Diagrams are all
%collected and grouped here in the Class Diagram.
\chapter{Discussion}
\label{sec:discussion}
\section*{Exception Handling}
The system uses both checked and unchecked exceptions based on the situation, correctly
reflecting the abstraction level for the exceptions.
For instance, \mintinline{java}{ItemNotFoundException} is used when an item is not found in the item registry,
appropriately named after the error condition and provides enough information about the error.
Moreover, exceptions like
\mintinline{java}{ItemRegistryException} and
\mintinline{java}{OperationFailedException} are tested,
satisfying the requirement of writing unit tests for the exception handling.

The system also includes javadoc comments, follows the principle of an object not changing state when an exception is thrown,
and notifies both users and developers.

\section*{Refactoring to Patterns}
I have tried to refactor to several design patterns, with the purpose of enhancing readability, modularity, and maintainability.
One can find a variety of Creational, Structural, and Behavioral Patterns as per the GoF design patterns.
The Singleton pattern is used for maintaining single logger instances and registries,
Facade pattern is used to provide a simplified interface to the registries,
and the Observer pattern is used to update sale and revenue displays.
The template method is used to avoid duplicated code.
The Factory method, Strategy and Composite pattern are all used for applying VAT, Discounts and promotions.

The explanation and motivation for using design patterns are not always detailed in the text.
I mention the use of various patterns but do not always provide the rationale for choosing those patterns.

\subsubsection*{The observers and their observables}
The system seems to have implemented the Observer pattern correctly with suitable objects.
The \mintinline{java}{Sale} and \mintinline{java}{CashRegister} objects are observed while the observer objects
are those that display sale details and total revenue.

I claim that this is a logical choice,
as any change in a sale or the cash register would require updating the displayed information.

The observers are only exposed to certain methods or attributes,
here I am implying that encapsulation is maintained, which promotes high cohesion and low coupling.

The data passed from the observed object (\mintinline{java}{Sale}) to observers is the sale details,
including the shopping cart details.
It's passed through an interface (\mintinline{java}{LimitedSaleView}) which ensures that encapsulation is not broken,
as it exposes only certain Sale methods.
Therefore, it maintains the state of the observed object while still allowing the observer to update its display.

\section*{Pattern Happy?}
The patterns are implemented according to the GoF (Gang of Four) design patterns.
The system implements Singleton, Abstract Factory, Factory method,
Composite, Facade, Observer, Strategy, and Template Method.

Probably an overuse of patterns!
I became quite ''Pattern-Happy'', eager to use as many patterns I could possibly comprehend.
\appendix
\chapter{UML -- The Refined Design}
\subsection*{System Operations}
\subsubsection*{The Start up, Main System Operation}
\begin{figure}[H]
    \begin{center}
        \includegraphics[width=\textwidth]{../../uml/output/uml_v3_004}
%        \includegraphics[trim=0cm 0cm 0cm 0cm, clip, width=\textwidth]{uml/output/uml_v3.png}
%        \caption{The Start Up \verb|Main| system operations}
        \caption{The Main System operation}
        \label{fig:start-up}
    \end{center}
\end{figure}
%
\subsubsection*{The StartSale System Operations}
\begin{figure}[H]
    \begin{center}
        \includegraphics[width=\textwidth]{../../uml/output/uml_v3_005.png}
%        \includegraphics[trim=0cm 0cm 51cm 0cm ,clip, width=\textwidth]{../uml/output/uml_v3.png}
%        \caption{The \verb|startSale| system operation }
        \caption{The Start Sale System Operation}
        \label{fig:start-sale}
    \end{center}
\end{figure}
%%
%\newpage
\subsubsection*{The registerItem System Operations}
\begin{figure}[H]
    \begin{center}
        \includegraphics[width=\textwidth]{../../uml/output/uml_v3_006.png}
%        \includegraphics[trim=0cm 0cm 40cm 0cm, clip, width=\textwidth]{uml/output/uml_v3.png}
%        \caption{The \verb|registerItem| system operation }
        \caption{The register item operation, including checked and unchecked exceptions}
        \label{fig:reg-item}
    \end{center}
\end{figure}

\begin{figure}[H]
    \begin{center}
        \includegraphics[width=\textwidth]{../../uml/output/uml_v3_007.png}
%        \includegraphics[trim=0cm 0cm 40cm 0cm, clip, width=\textwidth]{uml/output/uml_v3.png}
%        \caption{The \verb|registerItem| system operation }
        \caption{Item not previously found in shopping cart}
        \label{fig:item-not-in-shopping-cart}
    \end{center}
\end{figure}

\begin{figure}[H]
    \begin{center}
        \includegraphics[width=\textwidth]{../../uml/output/uml_v3_008.png}
%        \includegraphics[trim=0cm 0cm 40cm 0cm, clip, width=\textwidth]{uml/output/uml_v3.png}
%        \caption{The \verb|registerItem| system operation }
        \caption{Notify sale observers}
        \label{fig:notify-sale-observers}
    \end{center}
\end{figure}

\begin{figure}[H]
    \begin{center}
        \includegraphics[width=\textwidth]{../../uml/output/uml_v3_009.png}
%        \includegraphics[trim=0cm 0cm 40cm 0cm, clip, width=\textwidth]{uml/output/uml_v3.png}
%        \caption{The \verb|registerItem| system operation }
        \caption{Notify users and/or developers}
        \label{fig:notify-users-developers}
    \end{center}
\end{figure}

\subsubsection*{The endSale System Operations}
\begin{figure}[H]
    \begin{center}
        \includegraphics[width=0.5\textwidth]{../../uml/output/uml_v3_010}
%        \includegraphics[trim=0cm 0cm 50cm 0cm, clip, width=\textwidth]{uml/output/uml_v3.png}
%        \caption{The \verb|endSale| system operation }
        \caption{The endSale system operation}
        \label{fig:end-sale}
    \end{center}
\end{figure}
%
\subsubsection*{The payment operation, discount, bonuspoints and promotion}
\begin{figure}[H]
    \begin{center}
        \includegraphics[width=\textwidth]{../../uml/output/uml_v3_011}
%        \includegraphics[trim=0cm 0cm 30cm 0cm, clip, width=\textwidth]{uml/output/uml_v3.png}
        \caption{Register customer to sale.}
        \label{fig:reg-customer}
    \end{center}
\end{figure}

\begin{figure}[H]
    \begin{center}
        \includegraphics[width=\textwidth]{../../uml/output/uml_v3_012}
%        \includegraphics[trim=0cm 0cm 0cm 0cm, clip, width=\textwidth]{uml/output/uml_v3.png}
%        \caption{The \verb|addPayment| system operation }
        \caption{The payment system operation}
        \label{fig:pay}
    \end{center}
\end{figure}

\begin{figure}[H]
    \begin{center}
        \includegraphics[width=\textwidth]{../../uml/output/uml_v3_013}
%        \includegraphics[trim=0cm 0cm 0cm 0cm, clip, width=\textwidth]{uml/output/uml_v3.png}
%        \caption{The \verb|addPayment| system operation }
        \caption{Pricing strategy}
        \label{fig:pricing}
    \end{center}
\end{figure}

\begin{figure}[H]
    \begin{center}
        \includegraphics[width=0.5\textwidth]{../../uml/output/uml_v3_014}
%        \includegraphics[trim=0cm 0cm 0cm 0cm, clip, width=\textwidth]{uml/output/uml_v3.png}
%        \caption{The \verb|addPayment| system operation }
        \caption{Notify cash register observers}
        \label{fig:notify-cashregister-observer}
    \end{center}
\end{figure}

\newpage
\subsection*{Class Diagram}
\begin{figure}[H]
    \begin{center}
        \includegraphics[width=\textwidth]{../../uml/output/uml_v3}
        \caption{The Class Diagram}
        \label{fig:CD}
    \end{center}
\end{figure}

\begin{figure}[H]
    \begin{center}
        \includegraphics[width=\textwidth]{../../uml/output/uml_v3_001}
        \caption{Discount and VAT Factory.
        Algorithms used for calculating total price with
        disounts or promotions and for
        calulating VAT for a specific item}
        \label{fig:discount-vat-factories}
    \end{center}
\end{figure}

\begin{figure}[H]
    \begin{center}
        \includegraphics[width=\textwidth]{../../uml/output/uml_v3_002}
        \caption{The facade provides a unified interface to a
        set of interfaces in a subsystem.
        It represents a service container (implemented as a Facade).
        This service container is responsible for instantiating all registers (external systems/databases) and provides
        a unified and simplified interface for the client classes.
        }
        \label{fig:facade}
    \end{center}
\end{figure}

\begin{figure}[H]
    \begin{center}
        \includegraphics[width=\textwidth]{../../uml/output/uml_v3_003}
        \caption{The DTOs (Data Transfer Objects)}
        \label{fig:DTO}
    \end{center}
\end{figure}

\newpage

\listoflistings % Now typeset the list
\printbibliography
%med natbib
%\bibliographystyle{unsrtnat} % referenser vancouver
\end{document}
