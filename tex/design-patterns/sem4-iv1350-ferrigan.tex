\documentclass[a4paper]{scrreprt}
\usepackage{fancyhdr}
\pagestyle{fancy}
\usepackage[english]{babel}
\usepackage[utf8]{inputenc}

\usepackage{graphicx}
\usepackage{url}
\usepackage{textcomp}
\usepackage{amsmath}
\usepackage{lastpage}
\usepackage{pgf}
\usepackage{wrapfig}
\usepackage{fancyvrb}

%\usepackage[style=ieee]{biblatex} % om du vill köra ieee (tror leif kör den)
\usepackage[backend=biber,style=vancouver]{biblatex}
\addbibresource{bibfile.bib}
%Alternativt med natbib
%\usepackage[numbers]{natbib} %referenser vancouver
%\usepackage[numbers,round]{natbib} %referenser vancouver
%\usepackage{hyperref}

\usepackage{pdflscape}
% \usepackage{lscape}

% Code highligting
% \usepackage{minted}
\usepackage[outputdir=output/tex]{minted} % iom min makefile

\usepackage{caption}
\usepackage{hyperref}
\newenvironment{longlisting}{\captionsetup{type=listing}}{}
% \renewcommand\listoflistingscaption{Källkod....}
\renewcommand\listoflistingscaption{List of source codes}

\usepackage{paralist} %Inline lists

% Create header and footer
\headheight 27pt
\pagestyle{fancyplain}
\lhead{\footnotesize{Object-Oriented Design, IV1350}}
\chead{\footnotesize{Seminar 3 Implementation}}
\rhead{}
\lfoot{}
\cfoot{\thepage\ (\pageref{LastPage})}
\rfoot{}

% Create title page
\title{Seminar 4, Exceptions and Design Patterns}
\subtitle{Object-Oriented Design, IV1350}
\author{Vincent Ferrigan ferrigan@kth.se}

\begin{document}

\maketitle

\tableofcontents %Generates the TOC

\chapter{Introduction}
The purpose of this report is to discuss the implementation of exception handling
and design patterns.
The aim of the fourth seminar was on practicing the design
and coding of exception handling, polymorphism, and design patterns,
specifically as outlined in chapters eight and nine of the course book.

The seminar consisted of two tasks.
The first task was to use both checked and
unchecked exception handling.
The former to manage the alternative flow of $3-4a$
in the Process Sale, and the latter to indicate database failure (such as a
database server not running or database connectivity lost).
The latter had to
be simulated.

Additional exception handling were implemented;
\begin{enumerate}
    \item Exceptions thrown by reflective operations (issues related to the Factory Design Pattern see ****%TODO),
    \item I/O exceptions (related to reading and writing to the flat file databases and creating and writing to the revenue and error logs),
    \item \mintinline{java}{IllegalStateException}, which is thrown if system operations are called in a wrong order.
    For example when registering an item before initializing a sale or calling the pay operation before ending sale
    when methods are called in wrong order.
\end{enumerate}

The second task consisted of two parts.
The first part, \emph{task 2a}, revolved
around employing the Observer Design Pattern to introduce a new functionality:
displaying the total revenue from the start of the program, each time it was
updated.
This functionality was implemented through the creation of two new
classes.
The first called \mintinline{java}{TotalRevenueView}, responsible for displaying the
total income and a second class called \mintinline{java}{TotalRevenueFileOutput} for logging the
income to a file.

The second part of the second task, \emph{task 2b}, involved refactoring other parts of
the code to two more GoF patterns.
Here the author chose to implement
\begin{enumerate}
\item \textbf{The Creational Patterns}
\begin{enumerate}
    \item \emph{Singleton},
    \item \emph{Abstract Factory} and
    \item \emph{Factory method}.
\end{enumerate}
\item \textbf{The Structural Patterns}
\begin{enumerate}
    \setcounter{enumi}{3}
    \item \emph{Composite} and
    \item \emph{Facade}.
\end{enumerate}
\item \textbf{The Behavioral Patterns}
\begin{enumerate}
    \setcounter{enumi}{5}
    \item \emph{Observer},
    \item \emph{Strategy} and
    \item \emph{Template Method}.
\end{enumerate}
\end{enumerate}

%\textbf{Creational Patterns}
%\begin{inparaenum}[1)]
%\item \emph{Singleton},
%\item \emph{Abstract Factory} and
%\item \emph{Factory method},
%the \textbf{Structural Patterns}
%\item \emph{Composite} and
%\item \emph{Facade} and the \textbf{Behavioral Patterns}
%\item \emph{Observer},
%\item \emph{Strategy} and
%\item \emph{Template Method}.
%\end{inparaenum}



%TODO Lägg till github konto.


\chapter{Method}
\section*{Tools}
This project was implemented in \emph{Java}.
The UML modeling tool used was \emph{PlantUML}.
PlantUML is an open-source tool for creating UML diagrams using a text-based
syntax.
It was chosen to enable version control through git and to avoid using a
proprietary graphical editor.
This report was also written in plain text mode -- \LaTeX.

All code was written in \emph{IntelliJ IDEA} and the unit tests were written with the \emph{JUnit 5} test framework.
Quick-fixes and editing was, however, done in \emph{Vim}.

\section*{The overall Work-flow}
Incremental refactoring, one feature at a time.
Automatic tests were run to make sure that the changes didn't break the code.

To improve the design and implement certain design patterns,
one had to continuously analyze the code-base to identify code smells and
(re)valuate parts of the code that can be refactored to a
particular design pattern (or patterns).

As a result, the system design was continuously refined,
allowing for more clean and maintainable code and a greater ease to add new code.

\section*{The project's resources and the system's configuration and data}
Instead of hard-coding data in the integration classes,
flat file databases are used to store records.
These CSV-files store,
read and update data for accounting, inventory (a product catalog included), and customer details.

The filenames and their paths are all found in the project's \verb|Config.Properties|
file under \verb|src/main/resources|.
The class names of the objects a \emph{Factory} can create, and the filenames of all the loggers are also
configured in the same file.

The properties are added to the \verb|System Properties| -- a set-up based on
\href{https://docs.oracle.com/javase/tutorial/essential/environment/sysprop.html}{The Java\texttrademark{} Tutorials - System Properties}

If a developer is having trouble loading the resource file \verb|config.properties|,
they will have to first check that the \verb|src/main/resources|
is correctly configured as a resource directory in your IDE.

Since the system reads and writes to files, handling exceptions became necessary.

\section*{Tests}
Tests are created using the JUnit 5 framework and follow best practice, i.e., they
have the same directory architecture as the program, and are placed outside the source
folder for the program (the SUT).

Similar to the SUT, the tests have their own Config Properties file, error-log file and flat file databases
(see listing ~\ref{listing:test-config.properties}) and are added to the \verb|System Properties| with
a \mintinline{java}{@BeforeAll} method (see listing ~\ref{listing:before-all-tests-setup}).

In this way, the inventory can be changed on the fly, just by manipulating
the \verb|inventory_items.csv| file which is placed \verb|test-data/db| directory.
The test can create, compare and manipulate its own database and log-files.
This enables testing classes in both the integration and util layer package.
. %TYP TODO men har du gjort så? Lägg till ErrorFileLogHandler test, TotalRevenueFileOutput test.

\begin{longlisting}
    \inputminted[
        label=@BeforeAll tests setup,
        linenos=true,
        bgcolor=lightgray,
        firstline=1,
        lastline=25,
%        frame=single,
        fontsize=\footnotesize,
    ]{java}{../../src/test/resources/config.properties}
    \caption{The Config Properties file for the JUnit tests.}
    \label{listing:test-config.properties}
\end{longlisting}

\begin{longlisting}
    \inputminted[
        label=@BeforeAll tests setup,
        linenos=true,
        bgcolor=lightgray,
        firstline=10,
        lastline=34,
%        frame=single,
        fontsize=\footnotesize,
    ]{java}{../../src/test/se/kth/iv1350/POSTestSuperClass.java}
    \caption{The Config Properties are added to the System Properties in a @BeforeAll setup method.
    This method is placed in a superclass, which is extended in all test that require a pre-configruation}
    \label{listing:before-all-tests-setup}
\end{longlisting}

\section*{Refactoring to Patterns}

\subsubsection{Discount and promotion algorithms}
Conditional discount logic can be replaced with a combination of the Strategy, Factory and Composite Pattern.
Here, each discount (or promotion) algorithm
can be defined in a separate strategy-class, but with a common interface.
see listing 1 ~\ref{listing:discount-strategy}.

\begin{longlisting}
    \inputminted[
        label=The common Discount-Strategy Interface,
        linenos=true,
        bgcolor=lightgray,
        firstline=6,
        lastline=20,
%        frame=single,
        fontsize=\footnotesize,
    ]{java}{../../src/main/se/kth/iv1350/integration/pricing/DiscountStrategy.java}
    \caption{Caption text av något slag.}
    \label{listing:discount-strategy}
\end{longlisting}

Knowing, however, which discount to choose from requires a third pattern.....
The need to substitute or add discounts at runtime required....
Creational knowledge is also moved to the factory, since discounts will be substituted or added at runtime (and after deployment )

and some duplicated code was eliminated by using the Template Method Pattern.


%a single instance available to all of the program, for example, a single database object shared by different parts of the program.g

\newpage
\chapter{Result}
\label{sec:result}
xxxxx

The entire program can be found here:

\subsubsection*{GitHub-Repo}
\url{https://github.com/VincentFerrigan/kth-iv1350-point-of-sale}

%\section{The Class Diagram}
%\begin{figure}[h!]
%    \begin{center}
%        \includegraphics[width=\textwidth]{uml/output/CD_Sem1}
%        \caption{The class diagram after all system operations have been designed.}
%        \label{fig:CD}
%    \end{center}
%\end{figure}
%The Class Diagram illustrated in figure \ref{fig:CD} emphasizes the links between the objects, i.e.\ the instances of
%our classes.
%It also groups the classes into subsystems in accordance with the MVC architectural pattern.
%
%All possible attributes and methods that belong to a class (including their parameters) are shown.
%In other words, all methods (and parameters) that were depicted in the Interaction Diagrams are all
%collected and grouped here in the Class Diagram.
%\newpage
%%\begin{landscape}
%    \section*{The Refined Design}
%    \subsection*{Class Diagram}
%%    \newgeometry{left=0cm,right=0cm,top=0cm,bottom=0cm}
%
%    \begin{figure}[h!]
%        \begin{center}
%            \includegraphics[width=\textwidth]{../../uml/output/uml_v3}
%            \caption{The Class Diagram}
%            \label{fig:CD}
%        \end{center}
%    \end{figure}
%
%    \begin{figure}[h!]
%        \begin{center}
%            \includegraphics[width=\textwidth]{../../uml/output/uml_v3_001}
%            \caption{The DTOs (Data Transfer Objects)}
%            \label{fig:DTO}
%        \end{center}
%    \end{figure}
%
%\newpage
%\subsection*{System Operations}
%\subsubsection*{The Start up, Main System Operation}
%\begin{figure}[h!]
%    \begin{center}
%        \includegraphics[width=\textwidth]{../../uml/output/uml_v3_002}
%%        \includegraphics[trim=0cm 0cm 0cm 0cm, clip, width=\textwidth]{uml/output/uml_v3.png}
%%        \caption{The Start Up \verb|Main| system operations}
%        \caption{The Start Up Main system operations}
%        \label{fig:SD5}
%    \end{center}
%\end{figure}
%
%\subsubsection*{The StartSale System Operations}
%\begin{figure}[h!]
%    \begin{center}
%        \includegraphics[width=\textwidth]{../../uml/output/uml_v3_003}
%%        \includegraphics[trim=0cm 0cm 51cm 0cm ,clip, width=\textwidth]{../uml/output/uml_v3.png}
%%        \caption{The \verb|startSale| system operation }
%        \caption{The startSale system operation }
%        \label{fig:SD}
%    \end{center}
%\end{figure}
%%
%\newpage
%\subsubsection*{The registerItem System Operations}
%\begin{figure}[h!]
%    \begin{center}
%        \includegraphics[width=\textwidth]{../../uml/output/uml_v3_004}
%%        \includegraphics[trim=0cm 0cm 40cm 0cm, clip, width=\textwidth]{uml/output/uml_v3.png}
%%        \caption{The \verb|registerItem| system operation }
%        \caption{The registerItem operation }
%        \label{fig:SD1}
%    \end{center}
%\end{figure}
%
%\subsection*{Displaying the output of current sale}
%\begin{longlisting}
%    \inputminted[
%        label= Output of a current sale,
%        linenos=false,
%%        bgcolor=lightgray,
%        firstline=1,
%        lastline=8,
%%        frame=single,
%        fontsize=\footnotesize,
%    ]{bash}{output/outputs.txt}
%    \caption{The Receipt printed to System.out}
%    \label{listing:checkoutOutput}
%\end{longlisting}
%
%\subsubsection*{The endSale System Operations}
%\begin{figure}[h!]
%    \begin{center}
%        \includegraphics[width=\textwidth]{../../uml/output/uml_v3_005}
%%        \includegraphics[trim=0cm 0cm 50cm 0cm, clip, width=\textwidth]{uml/output/uml_v3.png}
%%        \caption{The \verb|endSale| system operation }
%        \caption{The endSale system operation }
%        \label{fig:SD2}
%    \end{center}
%\end{figure}
%
%\subsubsection*{End of Sale - Displaying the output of a checkout}
%\begin{longlisting}
%    \inputminted[
%        label= Output of a checkout,
%        linenos=false,
%%        bgcolor=lightgray,
%        firstline=61,
%        lastline=72,
%%        frame=single,
%        fontsize=\footnotesize,
%    ]{bash}{output/outputs.txt}
%    \caption{The checkout shopping cart printed to System.out}
%    \label{listing:checkout}
%\end{longlisting}
%
%\subsubsection*{The discountRequest System Operations}
%\begin{figure}[h!]
%    \begin{center}
%        \includegraphics[width=\textwidth]{../../uml/output/uml_v3_006}
%%        \includegraphics[trim=0cm 0cm 30cm 0cm, clip, width=\textwidth]{uml/output/uml_v3.png}
%        \caption{The discoutRequest system operation }
%        \label{fig:SD3}
%    \end{center}
%\end{figure}
%
%\subsubsection*{The addPayment System Operations}
%\begin{figure}[h!]
%    \begin{center}
%        \includegraphics[width=\textwidth]{../../uml/output/uml_v3_007}
%%        \includegraphics[trim=0cm 0cm 0cm 0cm, clip, width=\textwidth]{uml/output/uml_v3.png}
%%        \caption{The \verb|addPayment| system operation }
%        \caption{The addPayment system operation }
%        \label{fig:SD4}
%    \end{center}
%\end{figure}
%
%\subsubsection*{Receipt}
%\begin{longlisting}
%    \inputminted[
%        label= Receipt,
%        linenos=false,
%%        bgcolor=lightgray,
%        firstline=73,
%        lastline=91,
%%        frame=single,
%        fontsize=\footnotesize,
%    ]{bash}{output/outputs.txt}
%    \caption{The Receipt printed to System.out}
%    \label{listing:receipt}
%\end{longlisting}
%
%%    \restoregeometry
%%\end{landscape}
%

\chapter{Discussion}
\label{sec:discussion}

\listoflistings % Now typeset the list
\printbibliography
%med natbib
%\bibliographystyle{unsrtnat} % referenser vancouver
\end{document}
